%Author: Henry Budris
%Updated: 04/15/2020

\documentclass[12]{article}
\usepackage[left=0.75in, right=0.75in, top=0.75in, bottom=0.75in]{geometry}
\usepackage{amsmath,amsthm,amscd,amssymb,mathrsfs}
\usepackage{enumitem, xstring}

%This document contains many of the custom commands I use to write my assignments. 
%They make typing sometimes tedious characters to type repeatedly (ie. a {) easier as well 
%as automatically setting up environments.  The custom commands are divided into named 
%sections.  Each section has directions on how to use the commands, what mode you must use the
%command in (ie. commands requiring math mode are surrounded in $), and what parameters the
%command takes.

%NOTE: When compiling your document, compile on your main.tex page, not this one!  Compiling on this page will cause errors.

%Commands for document headers
%Use: \command{assignment number}{due date}{author(s)}
\newcommand{\hwhead}[3]{
\begin{center} 
    \textbf{MATH 390, Homework #1} \n 
    \textit{Due date: #2} \n 
    Author: #3 \n 
\end{center}}

\newcommand{\labhead}[4]{
\begin{center}
    \textbf{MATH 390, Lab #1} \n 
    \textit{Due date: #2} \n 
    Authors: #3 \n 
    Team: #4 \n 
\end{center}}
%--------------------------------------------------------------

%Commands for formatting and inserting special characters
%Use: \command or $\command$
\newcommand{\n}{\\*}
\newcommand{\tb}{\null\quad}
\newcommand{\ob}{\{}
\newcommand{\cb}{\}}

%Contributor: Nick Andersen
\newcommand{\s}{\vskip2mm\vspace{2mm}}
\newcommand{\sd}{\vskip0mm\vspace{2mm}}
\newcommand{\sg}{\vskip0mm\vspace{3mm}}
\newcommand{\size}[2]{{\fontsize{#1}{0}\selectfont#2}}

%Commands for set notation
%Use: $\command{definition*}{rule*}$
\newcommand{\set}[1]{\ob #1 \cb}       %*
\newcommand{\tmod}{\text{mod}~}
\newcommand{\tand}{~\text{and}~}
\newcommand{\setz}{\mathbb{Z}}
\newcommand{\setn}{\mathbb{N}}
\newcommand{\setq}{\mathbb{Z}}
\newcommand{\setr}{\mathbb{R}}

%Commands for setting up a proof
%Use: <\comamnd{proof body}>
\theoremstyle{plain}
\newtheorem{theorem}{Theorem}
\newcommand{\pf}[1]{
\begin{proof} #1
\end{proof}
}

%Commands for ordered pair
\newcommand{\op}[2]{(#1, #2)}